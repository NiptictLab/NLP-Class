\documentclass[conference]{IEEEtran}
\IEEEoverridecommandlockouts
% The preceding line is only needed to identify funding in the first footnote. If that is unneeded, please comment it out.
\usepackage[table]{xcolor}
\usepackage{cite}
\usepackage{amsmath,amssymb,amsfonts}
\usepackage{algorithmic}
\usepackage{graphicx}
\usepackage{textcomp}
\usepackage{xcolor}
\usepackage{caption}
\usepackage{subcaption}
\usepackage{array}


\usepackage{mathtools} 
\usepackage{wrapfig}
\usepackage{verse}

\renewcommand{\arraystretch}{1.5}

\newcommand{\attrib}[1]{%
\nopagebreak{\raggedleft\footnotesize #1\par}}
\renewcommand{\poemtitlefont}{\normalfont\large\itshape\centering}

%added by Ye
\usepackage{fontspec}

%for typing Myanmar text, you can also used with Myanmar3 font
\newfontfamily {\padauktext}[Script=Myanmar]{Padauk}

%for double quote
\newcommand{\quotes}[1]{``#1''}
\renewcommand{\baselinestretch}{0.86}
%added by Ye

\def\BibTeX{{\rm B\kern-.05em{\sc i\kern-.025em b}\kern-.08em
    T\kern-.1667em\lower.7ex\hbox{E}\kern-.125emX}}
\begin{document}

\title{Tasty Foods in Myanmar}

\author{\IEEEauthorblockN{1\textsuperscript{st} Ei Thandar Phyu}
\IEEEauthorblockA{\textit{Information Science and Technology} \\
\textit{University of Technology (Yadanarpon Cyber City)}\\
Pyin Oo Lwin, Myanmar \\
eithandarphyu@utycc.edu.mm}
}

\maketitle

\begin{abstract}
For all tourists, Myanmar food is always one of the most wonderful things to taste and experience. As well as Indian and Chinese food, Burmese regional dishes will also be found and different states have their own signature flavours such as Shan or Mon curries. As Myanmar is a coastal country, a wide variety of seafood will be encountered in areas close to the sea. Due to the hot temperatures across Myanmar, preserved meat is usually served in the centre of the country.
Myanmar is now an opening country to the outside world, thus, tourists can possibly discover a cuisine. Myanmar food is said to be influenced in Southern and Southeast Asia countries. With a list of ingredients that are not found in any other cuisine, there is much to discover. Like most of Southeast Asia countries, restaurants and booths in Myanmar tend to specialize in a dish or style of food. In this review paper, Mote Lone Yay Paw (Rice Balls), Htamane, Laphet Thoke (Tea leaf salad) are described with their ingredients and preparation.\end{abstract}

\begin{IEEEkeywords}
Myanmar, Mote Lone Yay Paw (Rice Balls), Htamane, Laphet Thoke (Tea leaf salad)
\end{IEEEkeywords}

\section{Introduction}
Because Myanmar has diverse geographical features, favorable seasonal conditions and is naturally endowed with fertile soil and water resources, it boasts an abundant supply of food in a great variety all year around. Myanmar lies between two great and very different cultures which have influenced not only religion, culture and arts, but also the preparation of food. During the colonial period, the influx of Chinese and Indians also had an impact on Myanmar traditional food, introducing new items. With the advent of globalization and trade liberalization, most famous foods from around the world are available in the cities, yet the majority of Myanmar people still cherish their own food, ensuring that its essence and uniqueness remains unchanged \cite{b1}.

Myanmar people have a long tradition of preparing food in their own way and the history of traditional food may be as old as the culture and arts of its people. Lots of herbs and spices are used, and most dishes are highly flavoured.The best-known Burmese dish is On-no-khaukswe, a chicken curry based on coconut milk, served with plain boiled rice or egg noodles. The other popular meals include laphet (tea leaf salad), Shan-style rice, Burmese curry, Burmese sweet snacks, Shan-style ‘tofu’ noodles, Nangyi thoke, Mohinga, Shan-style noodles \cite{b1}. Southern Myanmar, particularly the area around Mawlamyine is known for its cuisine, as the Burmese proverb goes: "Mandalay for eloquence, Mawlamyine for food, Yangon for boasting" \padauktext{(မန္တလေးစကား ရန်ကုန်အကြွား မော်လမြိုင်အစား)} \cite{b2}.
	
\section{Mote Lone Yay Paw (Rice Balls)}
\label{sec:MoteLoneYayPaw}
Mote Lone Yay Paw (Burmese: \padauktext{မုန့်လုံးရေပေါ်}) is a traditional sweet rice balls. The name says food that float on water, in Myanmar language. It is usually stuffed with palm sugar. This sweet snack is often served during the celebration of Myanmar New Year which is also know as Thingyan Festival(\padauktext {သင်္ကြန်ပွဲတော်}), in mid-April to mark the lunar new year \cite{b3}. People, young and old, throw water over each other to celebrate and welcome the new year by washing away bad luck. Celebration of the Thingyan Festival is incomplete without this seasonal snack. It’s also a time when family and friends get together and the kitchen is bursting with aunties making rice dumplings and other elaborate meals \cite{b4}. During the festival, which includes throwing water at each other, some youngsters, in sprite of teasing each other, stuff hot chillies in rice balls instead of palm sugar. Its enjoyable for children or quite bonding moment with a stranger \cite{b3}.

\begin{figure}[ht!]
  \centering
\includegraphics[width=0.33\textwidth]{./fig/MLYP1.jpg}
  \caption{Mote Lone Yay Paw}
\label{fig:MLYPFig1}
\end{figure}
 
In Myeik, the southern region of Myanmar, rice balls is served with coconut dressing. It is usually stuffed with palm sugar and peanut. The receipe is similar to another Myanmar food, Mote Lone Yay Paw, but it is often served with pearl sago and coconut dressing. So this is a perfect receipe to make and share during Myanmar New Year as a variation to a more popular food Mote Lone Yay Paw \cite{b3}.

\begin{figure}[ht!]
  \centering
\includegraphics[width=0.33\textwidth]{./fig/MLYP2.jpg}
  \caption{Mote Lone Yay Paw with coconut dressing}
\label{fig:MLYPFig2}
\end{figure}

\subsection{Ingredients}
\label{subsec:MLYPIngredient}
\begin{table}[h!]
\caption{\label{table:MLYPIng} Ingredient of Mote Lone Yay Paw}
\begin{center} 
\begin{tabular}{ |c|c| }
 \hline
 glutinous rice flour(\padauktext{ကောက်ညှင်းမှုန့်}) & 800 g\\[5pt] 
 \hline
 rice flour(\padauktext{ဆန်မှုန့်}) & 200 g\\[5pt]
 \hline
 coconut(\padauktext{အုန်းသီး}) & 1 (grated)\\[5pt]
 \hline
 water(\padauktext{ရေ}) & 1 cup\\[5pt]
 \hline
 plam sugar(\padauktext{ထန်းလျက်}) & 200 g (cut into cubes)\\[5pt]
 \hline
\end{tabular}
\end{center}
\end{table}

\subsection{Preparation}
\label{subsec:MLYPPreparation}
\begin{enumerate}
   \item Sift glutinous rice flour and rice flour in a large bowl and mix thoroughly. If you want to add color, you can add 1/2 teaspoon of apple flavor color.
   \item Add 1 cup of water in the mixture and knead well until the dough is smooth.
   \item Scoop small doughs, roll into balls and fatten.
   \item Place a chunk of palm sugar in the center of flat circled dough and put back dough into a ball shape.
   \item Boil water in a large pot. Once boiling, drop in the balls.
   \item The balls will float to the surface when cooked.
   \item Take out the balls and put them on banana leave or plate.
   \item Spread grated coconut on rice balls and serve hot \cite{b3}.
\end{enumerate}

\section{Htamane}
\label{sec:Htamane}
\begin{wrapfigure}{l}{0.22\textwidth}
  \centering
\includegraphics[width=0.2\textwidth]{./fig/HMN2.jpg}
\label{fig:HMNFig}
\end{wrapfigure}

Htamane (Burmese: \padauktext{ထမနဲ}) is a glutinous rice-based savory snack, and a seasonal festive delicacy in Myanmar. The traditional delicacy is ceremonially prepared around and on the full moon day of Tabodwe (\padauktext{တပို့တွဲ}​), the $11^th$ lunar month on the traditional Burmese calendar (roughly in February), just as the cool season ends \cite{b5}. In Myanmar literature it is said that yagu and htamane are two different names of the same Myanmar delicacy. But in fact they are two different kinds. While yagu is just a rice porridge with or without milk and sugar, htamane is a delicacy prepared with glutinous rice from the first harvest Seasoned with ingredients and condiments. Myanmar believes that every herb, leaf, roof, tuber, fruit has medicinal potency, giving rise to the epigram, ‘food is medicine, medicine is food’ (\padauktext{အစာလဲဆေး ဆေးလဲအစာ}). 

Treatises on traditional medicine ‘Myanmar Pharmacopeia’ mentions that Htamane should be taken in this month, as preventive medicine to generate warmth in the body to combat the inclemency of the biting cold weather \cite{b6}. According to Treatise on Myanmar traditional medicine, the cold in Tabodwe dries up body skin. There is no perspiration to moisten the body skin. So it is necessary to take a little more of vegetable oil to protect the unbearable cold of Tabodwe. Oily htamane is eaten as a sort of preventive medicine by Myanmar old folks. 

Some pagodas and monasteries, including the Shwedagon Pagoda, hold htamane-making competitions (\padauktext{ထမနဲထိုးပြိုင်ပွဲ})\cite{b5}. The making of Htamane is more of a fun activity which promotes team work, culture and traditions. The Htamane festival is held throughout the country every year to celebrate and honor the achievement of the monsoon rice harvest. This Festival helps promote unity and mutual support. 

The tools required to make glutinous rice include a huge wide-rimmed iron bowl two long-handled stirring paddles three bricks for the makeshift fireplace and firewood.

\subsection{Ingredients}
\label{subsec:HMNIngredient}
\begin{description}
   \item Glutinous rice (\padauktext{ကောက်ညှင်းဆန်})
   \item Peanut oil (\padauktext{​မြေပဲဆီ})
   \item Coconut (\padauktext{အုန်းသီး})
   \item Peanuts (\padauktext{မြေပဲ})
   \item Ginger (\padauktext{ဂျင်း})
   \item Salt (\padauktext{ဆား})
\end{description}

\subsection{Preparation}
\label{subsec:HMNPreparation}
 \renewcommand{\labelitemi}{$\blacksquare$}
 \begin{itemize}
   \item  Coconut is sliced on the carpenter's plane.
 \end{itemize}
 \begin{itemize}
   \item  The sliced coconut and ginger are fried on the big wok.
 \end{itemize}
\begin{itemize}
   \item  Oil and water are added in and stirring until fragrant
 \end{itemize}
\begin{itemize}
   \item  Glutinous rice and water are added in.
 \end{itemize}
\begin{itemize}
   \item  Women shelling the peanuts to remove the husk.
 \end{itemize}
\begin{itemize}
   \item  They grind the peanuts using an empty bottle.
 \end{itemize}
\begin{itemize}
   \item  The woman sprinkle the peanut and sesame seed mixture over the glutinous rice as the man mix in.
 \end{itemize}
\begin{itemize}
   \item  Four men use their strength to stir until the peanuts and sesame seeds mix in with the glutinous rice.The glutinous rice gets sticker as they get mixed \cite{b7}.
 \end{itemize}

\poemtitle{Poem for Htamane Preparation}
\settowidth{\versewidth}{\padauktext {တစ်စိတ်ဆန်သည်ကောက်ညှင်းရယ်နှင့်}}
\begin{verse}[\versewidth]
\padauktext {ထမနဲ ထိုးရန်ဖို့ စာချိုးလို့မှာခဲ့မယ် ...}\\
\padauktext {တစ်စိတ်ဆန်သည်ကောက်ညှင်းရယ်နှင့်}\\
\padauktext {တစ်ပိဿာဆီအမှန်သွင်းပါလို့}\\
\padauktext {ချင်း( ဂျင်း) က သုံးဆယ်}\\ 
\padauktext {မြေပဲဖြူလှော်တစ်ခွက်ကယ်နှင့်}\\
\padauktext {အုန်းသီးဆန်အတူဖတ်ပါလို့}\\
\padauktext {နှမ်းမပြတ်ဤနည်းနှယ်}\\
\padauktext {ရေစင်ကြည်ပထည့်ရိုး}\\
\padauktext {လက်နှစ်သစ်သာပ}\\
\padauktext {တွက်စစ်ကာချိန်ဆယ်သားရယ်နှင့်}\\
\padauktext {ရောနှောကလူအများတို့ရယ်}\\
\padauktext {အားသစ်လို့ထိုး ...}\\
\end{verse}

\begin{figure}[ht!]
  \centering
  \begin{minipage}[b]{0.24\textwidth}
    \includegraphics[width=\textwidth]{./fig/HMN1.jpg}
    \caption{Htamane Ingredients}
  \end{minipage}
  %\hfill{0.1in}
  \begin{minipage}[b]{0.24\textwidth}
    \includegraphics[width=\textwidth]{./fig/HMN4.jpg}
    \caption{Htamane Preparation}
  \end{minipage}
\end{figure}


\section{Laphet Thoke (Tea leaf salad)}
\label{sec:TaphetThoke}
Tea leaves, or in Myanmar, laphet (Burmese: \padauktext{လက်ဖက်}), are important in our daily life. In our country, the elderly usually say ‘ of all the fruits, the mango is the best, of all the meat, the pork is the best and of all the leaves laphet is the best’ (\padauktext{အသီးမှာသရက် အသားမှာဝက် အရွက်မှာလက်ဖက်}). We use them in a lot of teas, such as Chinese tea, Indian tea and sweet tea, but we also use them in the famous pickled tea leaves salad called laphet thoke (\padauktext{လက်ဖက်သုပ်}). Historically, tea leaves in Myanmar were used as a peace symbol, and were offered to warring kingdoms after the settling of a dispute. For this, fermented tea leaves were either just exchanged or consumed in the presence of your enemy, more often then not in the form of a Burmese tea leaf salad.Nowadays, the salad is a popular dish across Myanmar society, eaten as a snack while on the go, as a meal in the late afternoon or offered as a friendly gesture to guests as a means to show good hospitality.

This salad is one of the most traditional Myanmar snacks and for Myanmar people it is also a symbol of generosity, sharing and loving, because the salad is usually made when the family is gathered or when you welcome guests. There are so many ways to make the salad, as there are many different types of marinated pickled tea leaves. This salad can be made in different flavours and in different styles – sweet, sour or spicy – it’s your choice, but it is a must to eat them with crispy fried butter beans, fried chickpeas, fried garlic, roasted sesame seeds, roasted peanuts and dried shrimps. Traditionally, we prepare the leaves by laying out, separately, one tea leaf after another on a big plate \cite{b8}.

\begin{figure}[ht!]
  \centering
\includegraphics[width=0.35\textwidth]{./fig/LPT2.jpg}
  \caption{Laphet Thoke}
\label{fig:LPTFig1}
\end{figure}

\begin{figure}[ht!]
  \centering
\includegraphics[width=0.3\textwidth]{./fig/LPT3.jpg}
  \caption{Tomato stuffed with laphet}
\label{fig:LPTFig2}
\end{figure}

\subsection{Ingredients for the pickled tea leaves}
\label{subsec:TLIngredient}
\begin{table}[h!]
\caption{\label{table:LPIng} Ingredient of pickled tea leaves}
\begin{center}
 \begin{tabular}{||c c||} 
 \hline
 plain tea leaves & 180-200g  \\ [0.5ex] 
 \hline
 \hline
 salt & 2 teaspoons  \\ [0.5ex] 
 \hline
\hline
 vegetable oil & ¼ cup  \\ [0.5ex] 
 \hline
\hline
 lime juice & 1½ tablespoon   \\ [0.5ex] 
 \hline
\hline
 garlic (diced) & 3 cloves  \\ [0.5ex] 
 \hline
\end{tabular}
\end{center}
\end{table}

\subsection{Preparation of the pickled tea leaves}
\label{subsec:LPPreparation}
{\rowcolors{1}{gray!80!white!50}{gray!70!white!40}
\begin{tabular}{l c }
Discard the harsh leaves and stalks. \\
Wash the tea leaves gently with warm water \\
Knead with generous amount of salt. \\
Then gently squeeze the leaves to get the bitter taste off. \\
Then mix them with salt, garlic, lime juice and vegetable oil. \\
Put the marinated tea leaves in a clean glass jar \\
Pour more oil until the leaves are immersed in the oil. \\
A day later you have homemade pickled tea leaves \cite{b8}.  \\
\end{tabular}
}

\subsection{Ingredients for the tea leaf salad}
\label{subsec:LPTIngredient}
There are quite a few different recipes out there, but usually the dish will consist of most of the following:

\renewcommand{\labelitemi}{$\textasteriskcentered$}
 \begin{itemize}
   \item  tea leaves (\padauktext {လက်ဖက်})
 \end{itemize}
 \begin{itemize}
   \item  crispy beans mixture (\padauktext {အ​ကြော်စုံ})
 \end{itemize}
\begin{itemize}
   \item  green chillis (\padauktext {ငရုတ်သီးစိမ်း})
 \end{itemize}
\begin{itemize}
   \item  garlic (\padauktext {ကြက်သွန်ဖြူ})
 \end{itemize}
\begin{itemize}
   \item  sweet corn (optional) (\padauktext {ပြောင်းဖူး})
 \end{itemize}
\begin{itemize}
   \item  cabbage (\padauktext {​ဂေါ်ဖီထုပ်})
 \end{itemize}
\begin{itemize}
   \item  dried shrimp (optional) (\padauktext {ပုစွန်ခြောက်})
 \end{itemize}
\begin{itemize}
   \item  tomatoes (\padauktext {ခရမ်းချဉ်သီး})
 \end{itemize}
\begin{itemize}
   \item  lime (\padauktext {သံပုရာသီး})
 \end{itemize}
\begin{itemize}
   \item  salt (\padauktext {ဆား})
 \end{itemize}
\begin{itemize}
   \item  fish sauce (\padauktext {ငံပြာရည်})
 \end{itemize}

\subsection{Preparation for the tea leaf salad}
\label{subsec:LPTPreparation}
Boil the sweet corn or cook the frozen ones until it is soft enough to eat. Saute the sweet corns with a teaspoon of peanut oil in a sauce pan and set aside. Roll the cabbage leaves in a row and slice them very finely. Slice the tomatoes and dice the green chillies. Peel the garlic and slice them finely as well. Use a big bowl and add the pickled tea leaves, the crispy double fried beans mixture, cabbage, tomatoes, sweet corns, garlic, green chillies, fish sauce, dried shrimp powder, lime juice and fried peanut oil. Mix well. If you prefer sourer or spicier or more salty, add more lime juice or more chili or more fish sauce. For the vegetarian, use salt rather than fish sauce. Serve with warm plain rice and jasmine tea or Shan tea \cite{b8}.

\begin{figure*}[ht!]
  \centering
\includegraphics[width=0.4\textwidth]{./fig/LPT1.jpg}
  \caption{Tea Leaf Salad}
\label{fig:LPTFig3}
\end{figure*}

\section{Conclusion}
\label{sec:conclusion}

Myanmar food culture is known for being a mix of different influences. If you are used to Indian and Chinese food then you will find a huge amount to enjoy here. When you come to Myanmar you can sample many snacks and curries and rice is a staple food all over the country. The local Burmese cuisine is known for having quite a strong flavour. You will find a number of great specialty dishes in Myanmar that you absolutely mustn't miss. In this paper, we described three of these tasty food you should try   

%\section*{Acknowledgment}

\begin{thebibliography}{00}

\bibitem{b1} https://www.bestpricevn.com/travel-guide/article-Myanmar-Traditional-Food-135.html
\bibitem{b2} Janssen, Peter (25 September 2012). "Good food in Rangoon, seriously". Yahoo! 7. Archived from the original on 27 September 2012. Retrieved 4 October 2012.
\bibitem{b3} http://www.wutyeefoodhouse.com/en/?p=102
\bibitem{b4} https://hsaba.com/recipes/new-year-rice-dumplings
\bibitem{b5} "ရွှေတိဂုံ ထမနဲထိုး ပြိင်ပွဲ". MRTV-4. January 2012. Retrieved 23 February 2012.
\bibitem{b6} https://www.today-myanmar.com/htamane/ 
\bibitem{b7} https://www.winnerinnmyanmar.com/post/2017/02/13/hta-ma-ne-a-myanmar-delicacy-1
\bibitem{b8} https://www.mmtimes.com/special-features/168-food-and-beverage/7429-impress-your-guests-with-myanmar-salads.html
\end{thebibliography}
\end{document}
